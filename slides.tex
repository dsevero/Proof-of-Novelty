\documentclass{beamer}
\usepackage[utf8]{inputenc}
\usepackage{fontawesome}
\usepackage{hyperref}

\usepackage{utopia} %font utopia imported
% \setbeamercovered{transparent}

\usetheme{Warsaw}
\input{definitions.tex}
% \usecolortheme{orchid}

%------------------------------------------------------------
%This block of code defines the information to appear in the
%Title page
\title[Proof of Novelty] %optional
{Proof of Novelty}

\subtitle{A distributed consensus mechanism for securing content novelty}

\author[\href{https://dsevero.com}{dsevero.com}] % (optional)
{Daniel Severo}

\institute % (optional)
{Independent Scientist}


\date[VDC 2019] % (optional)
{Virtual Design Challenge \\ The University of British Columbia \\ Vancouver - December 2019}

%End of title page configuration block
%------------------------------------------------------------



%------------------------------------------------------------
%The next block of commands puts the table of contents at the 
%beginning of each section and highlights the current section:

\AtBeginSection{
  \begin{frame}
    \frametitle{Table of Contents}
    \tableofcontents[currentsection]
  \end{frame}
}
%------------------------------------------------------------

\AtEndDocument{
    \begin{frame}
        \Huge Thank you!
        \large
        \begin{enumerate}
            \item[\faTwitter] \href{https://twitter.com/_dsevero}{\_dsevero}
            \item[\faGithub] \href{https://github.com/dsevero}{dsevero}
            \item[$\rightarrow$] \href{https://dsevero.com}{dsevero.com}
        \end{enumerate}
    \end{frame}
}


\begin{document}

%The next statement creates the title page.
\frame{\titlepage}


%---------------------------------------------------------
%This block of code is for the table of contents after
%the title page
\begin{frame}
\frametitle{Table of Contents}
\tableofcontents

Target audience: undergraduate students in STEM with a basic knowledge of Blockchain technology.
\end{frame}
%---------------------------------------------------------


\section{Motivation}
\begin{frame}{Motivation}
    \begin{block}{Motivational question}
        Somebody sends you a video, how do you know it is \emph{trustworthy}?
    \end{block}\pause
    
    \vfill
    
    \begin{itemize}
        \item Existing content can be manipulated for ill-usage.
    \end{itemize}
    
    \vfill
    
    \begin{figure}
        \visible<2>{
            \centering
            \includegraphics[width=4cm,keepaspectratio]{assets/trustworth-example-obama.png}
            \caption{Obama speech out of context.}
        }
    \end{figure}
    
\end{frame}

\section{Trustworthiness}
\begin{frame}{Trustworthiness}
    What does it mean to trust content? \pause Usually, we want \emph{novelty} and \emph{authenticity}.
    \vfill
    \pause
    \begin{block}{Authenticity}
        \pause
        \begin{itemize}
            \item Content is authentic when its origin is indisputable.\pause
            \item Can be reasonably solved with Digital Signatures.\pause
            \item Easy to check, since verifying signatures is fast.
        \end{itemize}
    \end{block}
    
    \vfill
    
    \pause
    \begin{block}{Novelty}
        \pause
        \begin{itemize}
            \item Subjective in nature, depends on the context.\pause
            \item Hard to check, requires comparing against archives.
        \end{itemize}
    \end{block}
\end{frame}

\section{Proof of Novelty}
\subsection{Overview}
\begin{frame}{Proof of Novelty}
    \begin{enumerate}
        \item[1] The owner of content $\content$ wishes to prove to us it is trustworthy.
        \item[2] A collection of similar content $\ContentSet$ exists and is secured on a blockchain with content-addressable hashes.
    \end{enumerate}
    \vfill
    \includegraphics[width=1.0\textwidth]{assets/pon-diagram-0.pdf}
\end{frame}

\begin{frame}{Proof of Novelty}
    \begin{enumerate}
        \item[3] The owner makes a transaction on the blockchain and receives a random subset of content-hashes $\sV \subseteq \ContentSet$.
        \item[4] The owner calculates the similarity of $\content$ with the elements of $\sV$, using a predefined similarity measure.
    \end{enumerate}
    \vfill
    \includegraphics[width=1.0\textwidth]{assets/pon-diagram-1.pdf}
\end{frame}

\begin{frame}{Proof of Novelty}
    \begin{enumerate}
        \item[5] The blockchain chooses a random committee of peers that verify a subset of the results, $\sF \subseteq \sV$.
    \end{enumerate}
    \vfill
    \includegraphics[width=1.0\textwidth]{assets/pon-diagram-2.pdf}
\end{frame}

\begin{frame}{Proof of Novelty}
    \begin{enumerate}
        \item[6] Consensus is reached by the committee regarding the legitimacy of the owner's calculations, and $c$ is accepted or rejected into $\ContentSet$.
    \end{enumerate}
    \vfill
    \includegraphics[width=1.0\textwidth]{assets/pon-diagram-3.pdf}
\end{frame}

\subsection{System details}
\begin{frame}{System details}
    \begin{block}{Creating $\sF$ through sortition}
        \emph{Cryptographic Sortition} can form committees without the need of interactions or exposure. Any peer holding a private key can verify and prove self-membership in the committee using a \emph{Verifiable Random Function}.
    \end{block}\pause
    \vfill
    \begin{block}{Similarity measure $s \colon \mathbb{C}^2 \rightarrow \mathbb{R}$}
        A similarity measure is any function that quantifies the degree of similarity between objects, such as a \emph{Neural Neural Network}.
    \end{block}\pause
    \vfill
    \begin{block}{Evaluating the owners submitted results}
        The committee must estimate if the similarity calculations between $c$ and elements of $\sV$ are legitimate, by observing only $\sF$.
    \end{block}
\end{frame}

\subsection{Probabilistic Guarantee of Novelty}
\begin{frame}{Probabilistic Guarantee of Novelty}
    If the owner's content is accepted, it provides a \emph{probabilistic guarantee} that their content is novel.
\end{frame}

% \begin{frame}{Proof of Novelty}
%     \begin{columns}
%         \begin{column}{0.5\textwidth}
%             \begin{enumerate}
%                 \item The owner of content $\content$ wishes to prove to us it is trustworthy.
%                 \item A collection of similar content $\ContentSet$ exists and is secured on the blockchain with content-addressable hashes.
%             \end{enumerate}
%         \end{column}
%         \begin{column}{0.5\textwidth}
%             \begin{center}
%                 \includegraphics[width=0.5\textwidth]{image1.jpg}
%             \end{center}
%         \end{column}
%     \end{columns}
% \end{frame}


\section{Take-away points}
\begin{frame}{Take-away points}
    \emph{Proof of Novelty} ...
    \begin{itemize}
        \item is a certificate issued by a decentralized and trustless system;\pause
        \item provides probabilistic guarantees of content novelty;\pause
        \item can scale by using cryptographic sortition;\pause
        \item defines similarity through similarity measures;\pause
        \item can be extended to any content type by switching the similarity measure.
    \end{itemize}
\end{frame}

\section{Open Questions and Future Work}
\begin{frame}{Open Questions and Future Work}
    \begin{itemize}
        \item Implementing and running experiments on Smart Contract blockchains such as Ethereum;\pause
        \item Applying decentralized and collaborative machine learning to progressively update a similarity function;\pause
        \item Investigating statistical hypothesis tests and guarantees
        for different content and similarity functions;\pause
        \item Properly choosing $\sV$ and $\sF$ to optimize statistical and computational performance.
    \end{itemize}
\end{frame}

% \begin{frame}{Frame Title}
% \begin{columns}
% \begin{column}{0.5\textwidth}
%   some text here some text here some text here some text here some text here
% \end{column}
% \begin{column}{0.5\textwidth}  %%<--- here
%     \begin{center}
%      \includegraphics[width=0.5\textwidth]{image1.jpg}
%      \end{center}
% \end{column}
% \end{columns}
% \end{frame}
%---------------------------------------------------------
%Changing visivility of the text
% \begin{frame}
% \frametitle{Sample frame title}
% This is a text in second frame. For the sake of showing an example.

% \begin{itemize}
%     \item<1-> Text visible on slide 1
%     \item<2-> Text visible on slide 2
%     \item<3> Text visible on slides 3
%     \item<4-> Text visible on slide 4
% \end{itemize}
% \end{frame}

%---------------------------------------------------------


%---------------------------------------------------------
% %Example of the \pause command
% \begin{frame}
% In this slide \pause

% the text will be partially visible \pause

% And finally everything will be there
% \end{frame}
%---------------------------------------------------------

% \section{Second section}

%---------------------------------------------------------
%Highlighting text
% \begin{frame}
% \frametitle{Sample frame title}

% In this slide, some important text will be
% \alert{highlighted} because it's important.
% Please, don't abuse it.

% \begin{block}{Remark}
% Sample text
% \end{block}

% \begin{alertblock}{Important theorem}
% Sample text in red box
% \end{alertblock}

% \begin{examples}
% Sample text in green box. The title of the block is ``Examples".
% \end{examples}
% \end{frame}
%---------------------------------------------------------


%---------------------------------------------------------
%Two columns
% \begin{frame}
% \frametitle{Two-column slide}

% \begin{columns}

% \column{0.5\textwidth}
% This is a text in first column.
% $$E=mc^2$$
% \begin{itemize}
% \item First item
% \item Second item
% \end{itemize}

% \column{0.5\textwidth}
% This text will be in the second column
% and on a second tought this is a nice looking
% layout in some cases.
% \end{columns}
% \end{frame}
% %---------------------------------------------------------


\end{document}